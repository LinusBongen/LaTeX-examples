\section{Code}

\begin{verbatim}
\usepackage{listings}
\usepackage{xcolor}

\definecolor{codegreen}{rgb}{0,0.6,0}
\definecolor{codegray}{rgb}{0.5,0.5,0.5}
\definecolor{codeorange}{rgb}{1,0.49,0}
\definecolor{backcolour}{rgb}{0.95,0.95,0.96}

\lstdefinestyle{mystyle}{
    backgroundcolor=\color{backcolour},   
    commentstyle=\color{codegray},
    keywordstyle=\color{codeorange},
    numberstyle=\tiny\color{codegray},
    stringstyle=\color{codegreen},
    basicstyle=\ttfamily\footnotesize,
    breakatwhitespace=false,         
    breaklines=true,                 
    captionpos=b,                    
    keepspaces=true,                 
    numbers=left,                    
    numbersep=5pt,                  
    showspaces=false,                
    showstringspaces=false,
    showtabs=false,                  
    tabsize=2,
    xleftmargin=10pt,
}

\lstset{style=mystyle}
\end{verbatim}



\begin{verbatim}
\begin{lstlisting}[language=Java, caption={Main-Methode},label={lst:java_example}, mathescape=true]
    public class Auto {
        public static void main(String[] args) {
            // ...Anweisung
        }
    }
\end{lstlisting}
\end{verbatim}

\begin{lstlisting}[language=Java, caption={Main-Methode},label={lst:java_example}, mathescape=true]
    public class Auto {
        public static void main(String[] args) {
            // ...Anweisung
        }
    }
\end{lstlisting}