\section{Header}

    \subsection{Allgemeiner Header}
    Hier wird der allgemeine Header erklärt, der auf allen Seiten gleich bleibt.

    \begin{tabular}{|p{0.4\textwidth}|p{0.6\textwidth}|}
        \hline
        \texttt{\textbackslash usepackage\{fancyhdr\}} & Paket für erweiterte Kopf- und Fußzeilenformatierungen \\
        \hline
    \end{tabular}

    \vspace{0.5cm}

    \begin{verbatim}
        \pagestyle{fancy}
        \fancyhf{} % Alle Kopf- und Fußzeilenfelder leeren
        \rhead{name} % Rechter Kopfzeilenbereich
        \lhead{Übung} % Linker Kopfzeilenbereich
        \cfoot{\thepage} % Zentrierte Fußzeile mit Seitenzahl    
    \end{verbatim}

    \subsection{Mehrere Header}

    \begin{verbatim}
    \fancypagestyle{header1}{%
        \fancyhf{}
        \fancyhead[R]{Linus}
        \fancyhead[L]{Matrikelnummer: XXXXX}
        \fancyhead[C]{Übung 1}
        \fancyfoot[C]{\thepage}
        \renewcommand{\headrulewidth}{0.4pt}
    \fancypagestyle{header2}{%
        \fancyhf{}
        \fancyhead[R]{Linus}
        \fancyhead[L]{Matrikelnummer: XXXXX}
        \fancyhead[C]{Übung 1}
        \fancyfoot[C]{\thepage}
        \renewcommand{\headrulewidth}{0.4pt}
    }
    % Header anwenden
    \pagestyle{header1}
    \pagestyle{header2}
    \end{verbatim}

    \subsection{Header zur Überschrift}

    \begin{verbatim}
    \pagestyle{scrheadings}
    \clearpairofpagestyles
    \automark{section}
    \ohead{\headmark} 
    \ofoot*{\pagemark}
    \KOMAoptions{headsepline=0.5pt} 
    \end{verbatim}

