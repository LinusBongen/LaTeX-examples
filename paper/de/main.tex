\documentclass[12pt,listsleft]{scrartcl}                                % Dokumentenklasse: Artikel mit 12pt Schriftgröße und Listen links ausgerichtet
\usepackage{scrlayer-scrpage}                                           % Kopf- und Fußzeilen
\usepackage[left=2cm, right=3cm, top=2cm, bottom=2.5cm]{geometry}       % Seitenränder
\usepackage{newtxtext}                                                  % Times-ähnliche Schrift für den Text
\usepackage{newtxmath}                                                  % Times-ähnliche Schrift für die Mathematik
\usepackage[T1]{fontenc}                                                % Zeichenkodierung für die Ausgabe
\usepackage[ngerman]{babel}                                             % Deutsche Sprachanpassungen
\usepackage{textcomp}                                                   % Zusätzliche Textsymbole
\usepackage[utf8]{inputenc}                                             % Eingabekodierung (UTF-8)
\usepackage{amsfonts}                                                   % Mathematische Schriftarten
\usepackage[style=german,german=quotes]{csquotes}                       % Anführungszeichen-Stil für Deutsch
\usepackage[backend=biber, style=numeric]{biblatex}                     % Bibliographie- und Zitationsstil
\usepackage{graphicx}                                                   % Grafiken einfügen
\usepackage{tocbasic}                                                   % Anpassungen für das Inhaltsverzeichnis
\usepackage{acronym}                                                    % Abkürzungsverzeichnis
\usepackage{amsmath}                                                    % Erweiterte mathematische Funktionen
\usepackage{hyperref}                                                   % hyperref-Paket laden, um Hyperlinks zu aktivieren
\usepackage{cleveref}                                                   % Paket zum intelligenten Referenzieren
\addbibresource{literatur.bib}                                          % Laden der Bibliographie-Datei

%%%% Kopfzeilen einrichten %%%%
\pagestyle{scrheadings}
\clearpairofpagestyles
\automark{section}
\ohead{\headmark}                                                       % Kapiteltitel auf der rechten Seite
\ofoot*{\pagemark}                                                      % Seitenzahl auf der rechten Fußzeile
\KOMAoptions{headsepline=0.5pt}                                         % Horizontale Linie unter der Kopfzeile
%%%%%%%%%%%%%%%%%%%%%%%%%%%%%%%

%%%%% Einstellungen für cleveref auf Deutsch setzen %%%%%
\addto\extrasngerman{%
  \renewcommand{\figurename}{Abbildung}%
  \renewcommand{\tablename}{Tabelle}%
  \crefname{figure}{Abbildung}{Abbildungen}%
  \Crefname{figure}{Abbildung}{Abbildungen}%
  \crefname{table}{Tabelle}{Tabellen}%
  \Crefname{table}{Tabelle}{Tabellen}%
}
%%%%%%%%%%%%%%%%%%%%%%%%%%%%%%%%%%%%%%%%%%%%%%%%%%%%%%%%%

\begin{document}

\pagenumbering{Roman}                                                   % Seitenzahlen in römischen Zahlen


\begin{titlepage}
    \begin{center}
        \vspace*{1cm}
        
        {\LARGE \textbf{Hochschule Name}}\\[0.5cm]
        {\Large \textbf{Institut Name}}\\[1.5cm]
        
        {\Huge \textbf{Titel der Arbeit}}\\[0.5cm]
        {\Large \textbf{Art der Arbeit}}\\[1.5cm]
        \vspace*{14cm}
    \end{center}

        \begin{tabular}{ll}
            \textbf{Name:} & \myAutor \\
            \textbf{Studiengang:} & \myStudiengang \\
            \textbf{Betreuer:} & \myBetreuer \\
            \textbf{Prüfer:} & \myBetreuer \\
            \textbf{Zweitprüfer:} & \myZweitpruefer \\
            \textbf{Abgabeort:} & \myLocation \\
            \textbf{Abgabedatum:} & \today
        \end{tabular}
    
\end{titlepage}                                                   % Einbinden der Titelseite

\clearpage

\tableofcontents                                                        % Inhaltsverzeichnis
\clearpage

\listoffigures                                                          % Abbildungsverzeichnis
\clearpage

\listoftables                                                           % Tabellenverzeichnis
\clearpage

\markboth{Glossar}{Glossar} % Setzt "Glossar" in beide Kopfzeilen
\section*{Glossar}                                                      % Glossar (Abkürzungsverzeichnis)
% abbreviations.tex

\begin{acronym}
    \acro{AI}{Artificial Intelligence}
    \acro{ML}{Machine Learning}
\end{acronym}


\clearpage

\pagenumbering{arabic}                                                  % Seitenzahlen in arabischen Zahlen

\justify                                                                % Aktiviere Blocksatz


\section{Einleitung}
In dieser Arbeit wird das Konzept der \ac{AI} untersucht.\\
Hier beginnt die Einleitung deines Dokuments \cite{musterbuch}.
\begin{align}
    f(x) &= ax^2 + bx + c \\
    g(x) &= \int_{0}^{x} f(t) dt
\end{align}                                                      % Einbindung der Einleitung
\clearpage
\section{Hauptteil}

In diesem Abschnitt ist der Hauptteil.                                                       % Einbindung des Hauptteils
\clearpage
\section{Schluss}
Hier endet dein Dokument mit dem Schluss.                                                         % Einbindung des Schlussteils

\clearpage

\printbibliography                                                      % Ausgabe des Literaturverzeichnisses

\end{document}
