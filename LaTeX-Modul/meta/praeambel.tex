\usepackage{lipsum}                                                     % Einfuegen von zufälligen text 
\usepackage[T1]{fontenc}                                                % Legt den Ausgabe-Zeichensatz fest.
\usepackage[utf8]{inputenc}                                             % erweiterter Unicode-Satz
\usepackage[ngerman]{babel}                                             % 
\usepackage{hyperref}                                                   %
\usepackage{a4wide}                                                     % 
\usepackage{cite}                                                       % 
\usepackage{graphicx}                                                   %
\usepackage{listings}
\usepackage{xcolor}

% Code 
\usepackage{listings} % Für Code-Darstellungen
\usepackage{xcolor}   % Für Farbdefinitionen

% Optionen für das listings-Paket
\lstset{
    basicstyle=\ttfamily, % Schriftart für den Code
    keywordstyle=\color{blue}, % Farbe für Schlüsselwörter
    commentstyle=\color{green!50!black}, % Farbe für Kommentare
    stringstyle=\color{red}, % Farbe für Strings
    escapeinside={(*@}{@*)}, % für mathescape
}



% komische angeben
\setlength{\parindent}{10pt}
\setlength{\parskip}{0.5\baselineskip}


