
\usepackage[T1]{fontenc}                                                % Zeichenkodierung für die Ausgabe
\usepackage[utf8]{inputenc}                                             % Europäischen Zeichen
\usepackage[ngerman]{babel}                                             % deutsche Silben etc. 
\usepackage[a4paper,
            left=2cm,           % 3 cm
            right=3cm,          % 2,5 cm
            top=2cm,            % 2 cm                                  % Seitenränder
            bottom=2.5cm        % 2 cm
            ]{geometry}
\usepackage{hyperref}                                                   % Einfueghen von referenzen
\usepackage{lipsum}                                                     % Einfuegen von zufälligen text 
\usepackage{acronym}                                                    % Abkürzungsverzeichnis
\usepackage[autocite=footnote,
    style=apa,
    backend=biber,
    bibencoding=utf8,
    bibwarn=true,
    date=iso,
    dateera=astronomical,
    doi=false,
    giveninits=false,                                                   % konfiguration von Bibliographie- und Zitationssystem
    isbn=true,
    maxcitenames=2,
    seconds=true,
    sortlocale=de
    url=true,
    urldate=iso,
    ]{biblatex}
\usepackage{graphicx}                                                   % Grafiken einfügen
\usepackage{amsmath}                                                    % Erweiterte mathematische Funktionen
\usepackage[toc]{glossaries}                                            % Glossar
\usepackage{lipsum}
\usepackage{enumitem}                                                   % auflistungen









\addbibresource{literature/literature.bib} 