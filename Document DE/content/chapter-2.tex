% content/Introduction.tex

\section{Überschrift} \label{sec:Section}

\lipsum[1]


\subsection{Unterüberschrift} \label{subsec:Subsection}

\highlight{highlighted text}

\highlightB{highlighted text with different color}

\enquote{this text is in quotes}

\begin{quote}
This is a block quote.
\end{quote}

\acrfull{tor} is an acronym.
\acrshort{tor} is another acronym.
\acrlong{tor} is yet another acronym.


\gls{maths} is a glossary entry.
\Gls{maths} is a glossary entry with capital first letter.
\glspl{maths} is a plural glossary entry.

\footnote{This is a footnote.}

\cite{nguyen_machine_2018} is a citation.

\zB This is an example.

\todo{This is a to-do note.}

\& 

\begin{figure}[h]
    \centering
    \includegraphics[width=0.4\textwidth]{images/image1.png}
    \caption{Abbildung 1: Beispielbild}
    \label{fig:bild}
\end{figure}

\begin{table}[h]
    \centering
    \begin{tabular}{|p{5cm}|p{5cm}|}
        \hline
        \textbf{Vorteile} & \textbf{Nachteile} \\
        \hline
        Vorteil 1 & Nachteil 1 \\
        Vorteil 2 & Nachteil 2 \\
        Vorteil 3 & Nachteil 3 \\
        \hline
    \end{tabular}
    \caption{Tabelle 1: Vorteile und Nachteile}
    \label{tab:Tabelle1}
\end{table}

Die \gls{sym:energy} kann verschiedene Formen annehmen.
Beim ersten Verwenden wird das Symbol erklärt: \gls{sym:energy}.
Beim zweiten Mal nur noch das Symbol: \gls{sym:energy}.\gls{sym:energy}
