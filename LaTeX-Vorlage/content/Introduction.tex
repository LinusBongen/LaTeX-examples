% content/Introduction.tex

\section{Section}

% Hinter \LaTeX \@ fehlt das Leerzeichen.

Dies -- und manchmal -- auch das.

Das kostet dann \$50.

% mit Argument
\textbf{Fetter Text!}
% Umschalter
\bfseries
Text text text
\normalfont % zurückschalten

Laut \cite{nguyen_machine_2018}

\myfigure{image.png}{Beispielbild}
\myfigure{sinus_plot.eps}{Vektor datei}

\begin{enumerate}
    \item Aufzählung
\end{enumerate}

\begin{itemize}
    \item Liste
\end{itemize}

\begin{equation}
    Mathe
\end{equation}

\begin{description}
    \item[Lexikonähnliche Aufzählung]
\end{description}


\begin{table}[h]
    \centering
    \begin{tabular}{|p{5cm}|p{5cm}|} 
        \hline
        \textbf{Vorteile} & \textbf{Nachteile} \\
        \hline
        Vorteil 1 & Nachteil 1 \\
        Vorteil 2 & Nachteil 2 \\
        Vorteil 3 & Nachteil 3 \\
        \hline
    \end{tabular}
    \caption{Vor- und Nachteile einer bestimmten Sache}
    \label{tab:beispiel}
\end{table}


The \Gls{latex} typesetting markup language is specially suitable 
for documents that include \gls{maths}. \Glspl{formula} are 
rendered properly an easily once one gets used to the commands.

Given a set of numbers, there are elementary methods to compute 
its \acrlong{gcd}, which is abbreviated \acrshort{gcd}. This 
process is similar to that used for the \acrfull{lcm}.\footnote{Eine andere Fußnote mit mehr Informationen.}

    \subsection{Subsection}
    \lipsum[1]
    \ref{tab:beispiel}
    
        \subsubsection{Subsubsection}
        \lipsum[1] Der Satz geht weiter\dots